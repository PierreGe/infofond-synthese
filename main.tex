\documentclass[a4paper]{article}

\usepackage[utf8]{inputenc}
\usepackage[T1]{fontenc}
\usepackage[french]{babel}
\usepackage{fullpage}
\usepackage{hyperref}
\usepackage{amsmath}
\usepackage{amssymb}
\usepackage{upgreek}
\usepackage{color}
\usepackage[]{algorithm2e}
\usepackage{stmaryrd}
\usepackage{graphicx}
\usepackage{float}

\title{
    INFO-F-302 - Informatique Fondamentale\\
    \small Synthèse 2014-2015
}
\author{Florentin \bsc{Hennecker}}
\date{}

\newtheorem{theorem}{Théorème}[section]

\begin{document}
\maketitle
\tableofcontents

\section{Logique propositionnelle}

  \subsection{Coupure}
  \textit{(Les notations sont plus précises dans le cours)}\\
  $$ C_1 = p_1 \lor ... \lor p_{i-1} \lor \textcolor{red}{p} \lor p_{i+1} \lor ... \lor p_n$$
  $$ C_1 = s_1 \lor ... \lor s_{i-1} \lor \textcolor{red}{\lnot p} \lor s_{i+1} \lor ... \lor s_n$$
  \'Etant donné deux clauses $C_1$ et $C_2$ qui ont une proposition commune $p$
  positive dans $C_1$ et négative dans $C_2$, la règle de coupure permet de déduire
  la clause : 
  $$ C_3 = p_1 \lor ... \lor p_{i-1} \lor p_{i+1} \lor ... \lor p_n
     \lor s_1 \lor ... \lor s_{i-1} \lor s_{i+1} \lor ... \lor s_n$$

  On note par $C_1, C_2 \vdash^c_p C_3$ le fait que $C_3$ est déductible de 
  $C_1, C_2$ par coupure sur la proposition $p$.

  \paragraph{Exemple 1} $ \lnot e \lor b \lor p, \lnot p \lor a \lor b \vdash^c_p \lnot e \lor a \lor $
  \paragraph{Exemple 2} $ \lnot p, p, \vdash^c_p \bot $ (clause vide)

  \begin{theorem}
  Si $C_1, C_2 \vdash^c_p C_3$, alors $C_3$ est une conséquence logique de $C_1 \land C_2$
  \end{theorem}

  \subsubsection{Preuve par coupure}
  Une preuve par coupure est la déduction d'une clause à partir d'autres clauses uniquement
  en utilisant des coupures. On note par $S \vdash^c C$ le fait que $C$ est déductible de $S$ par coupure.

  \subsubsection{Réfutation}
  Une \textit{réfutation} d'un ensemble $S$ de clauses par coupure est une dérivation
  de la clause vide à partir de $S$.

  \begin{theorem}
  Si il existe une réfutation de $S$ par coupure alors l'ensemble de clauses $S$
  est non satisfaisable.
  \end{theorem}


% ---------------------------------------------------------------------------- %
\section{Le problème SAT}

  Un problème SAT a en entrée un ensemble de clauses $S$ et en sortie la réponse à 
  la question : "Est-ce que $S$ est satisfaisable?". On aimerait que l'algorithme
  retourne une valutation $V$ qui satisfait $S$ au cas où $S$ est satisfaisable.
  Un solveur Sat est un programme qui décide le  problème SAT. Ces solveurs
  ont une complexité dans le pire cas exponentielle.

  \paragraph{Motivations}
  \begin{itemize}
    \item beaucoup de problèmes s'expriment naturellement par des formules en FNC.
    \item les problèmes de la classe NP se réduisent tous au problème SAT en temps polynomial
    \item on peut donc écrire un bon algo pour SAT plutôt qu'un bon algo pour chacun de ces problèmes
  \end{itemize}

  \subsection{Variantes de SAT}

    \paragraph{2-SAT} les clauses ne contiennent qu'au plus deux littéraux. \textbf{Solvable en temps polynomial}
    
    \paragraph{QSAT} 
    décider la satisfaisabilité de formules de la forme $\forall p_1 \exists q_1 ... \forall p_n \exists q_n.\phi$
    où $\phi$ est une formule en CNF construite sur les propositions $p1, q_1, ... p_n, q_n$

    \paragraph{MAX-SAT} étant donnée une formule $\phi$ en CNF et un entier $k \in \mathbb{N}$, 
    peut-on satisfaire au moins $k$ clauses?

    \paragraph{WEIGHTED-MAX-SAT} on attribue des poids à chaque clause, on se donne
    un entier $r$ et on veut savoir si on peut satisfaire un ensemble de clauses dont la 
    somme des poids est au moins $r$.

  \subsection{Interprétation partielle}
  Une \textit{interprétation partielle} est un assignement noté $x/1$ ou $x/0$, 
  qui signifie qu'on assigne la valeur $1$ à $x$ ou la valeur $0$. Cela permet
  de simplifier les formules. 

    \paragraph{Exemple 1} : la formule $(x \lor y) \land (\lnot x \lor z \lor \lnot y)$ se simplifie
    en $z \lor \lnot y $ sous l'interprétation partielle $x/1$.\\

  On peut appliquer deux interprétations partielles à la formule $\phi$, ce qui se
  note ainsi : $\phi[x/b][y/b']$

    \paragraph{Exemple 2} Soit $a = x \lor y \lor z$ et $b = x \lor \lnot y \lor \lnot z$. Alors :
    \begin{itemize}
      \item $a[x/1] = \top$, $b[x/1] = \top$, $(a \land b)[x/1] = \top$
      \item $a[x/0] = y \lor z$
      \item $b[x/0] = \lnot y \lor \lnot z$
      \item $(a \land b)[x/0][y/0] = z$
      \item ...
    \end{itemize}

  \subsection{Proposition pivot}
  L'algorithme DPLL va essayer des interprétations partielles, la proposition choisie
  est la \textit{proposition pivot}. Il va successivement essayer de mettre la 
  proposition est vrai ou à faux, et tester récursivement la satisfaisabilité de formules
  simplifiées obtenues. Le choix de la proposition pivot peut évidemment fortement
  influencer le résultat.

  \subsubsection{Premier critère de choix}
  DPLL choisit en priorité la proposition d'une \textbf{clause unitaire} (un seul littéral)
  comme proposition pivot. 

  \paragraph{Exemple} Dans $x \land (y \lor \lnot z)$, $x$ est unitaire et doit 
  obligatoirement être interprété par $1$ pour satisfaire la formule.

  \paragraph{Exemple de propagation de clauses unitaires}
  Prenons $$\phi = (x\lor y) \land \lnot y \land (\lnot x \lor y \lor \lnot z)$$
  Alors $$ \phi[y/0] = x \land (\lnot x \lor \lnot z) $$
  Ce qui donne la nouvelle clause unitaire $x$ qui impose $ x = 1 $ :
  $$ \phi[y/0][x/1] =  \lnot z$$ 
  Ce qui ne contient qu'une seule clause unitaire et on obtient finalement :
  $$ \phi[y/0][x/1][z/0] = \top $$
  Donc $\phi$ est satisfaisable avec l'interprétation $V(x) = 1$ et $V(y) = V(z) = 0$

  \subsubsection{Deuxième critère de choix}
  Le deuxième critère de choix se base sur les \textbf{propositions à polarité unique}.

  \paragraph{Exemple} Dans la formule 
  $$\phi = (x \lor \lnot y \lor z) \land (x \lor \lnot z) \land (y \lor z) \land (x \lor \lnot y) $$
  $x$ apparaît toujours positivement, on peut donc directement lui assigner la valeur $1$
  sans être obligé de tester la valeur $0$.

  \subsection{Algorithme DPLL}
  DPLL($\phi$) retourne \texttt{VRAI} si $\phi$ est satisfaisable.

  \begin{algorithm}[H]
    \uIf{$\phi = \top$}{retourner \texttt{VRAI}}
    \uElseIf{$\phi = \top$}{retourner \texttt{FAUX}}
    \uElseIf{$\phi$ contient une clause unitaire $x$}{retourner DPLL($\phi[x/1]$)}
    \uElseIf{$\phi$ contient une clause unitaire $\lnot x$}{retourner DPLL($\phi[x/0]$)}
    \uElseIf{$\phi$ contient une proposition $x$ de polarité toujours positive}{retourner DPLL($\phi[x/1]$)}
    \uElseIf{$\phi$ contient une proposition $x$ de polarité toujours négative}{retourner DPLL($\phi[x/0]$)}
    \Else{choisir une proposition $x$ au hasard et retourner (DPLL($\phi[x/0]$) ou DPLL($\phi[x/1]$)) }
  \end{algorithm}

  \subsection{Transformation de Tseitin}
  Parfois, on cherche un résoudre un problème qui ne s'exprime pas facilement en FNC.
  La transformation de Tseitin va ajouter des nouvelles variables et des équivalences.

    \paragraph{Exemple} Prenons $ \phi = (p \land q) \lor \lnot (q \lor r) $. Dans 
    la transformation de Tseitin, on remplace $p \land q$ par $x_1$ et $\lnot (q \lor r)$ par $x_2$.
    Il faut donc réécrire la formule comme ceci :
    $$ (x_1 \lor x_2) \land (x_1 \leftrightarrow (p \land q)) \land (x_2 \leftrightarrow \lnot (q \lor r) )$$
    Il reste encore à mettre les deux formules $(x_1 \leftrightarrow (p \land q))$ et $(x_2 \leftrightarrow \lnot (q \lor r))$ sous FNC.\\

  La technique de Tseitin sera particulièrement intéressante lorsqu'on devra mettre
  sous FNC des formules qui sont sous forme normale \textit{disjonctive}.

  \subsection{Autre exemple de rajout de variable intéressant}
  Supposons qu'on ait $n$ variables $x_0, x_1, ..., x_{n-1}$ et qu'on veuille
  exprimer qu'exactement une de ces variables doit être vraie. 

    \subsubsection{Solution naïve}
    On considère les deux formules suivantes en conjonction :
    \begin{itemize}
      \item \textbf{Au moins une :} $\bigvee^{n-1}_{i=0} x_i$
      \item \textbf{Au plus une :} $\bigwedge_{0 \leq i < j < n} \lnot x_i \lor \lnot x_j $
    \end{itemize}
    Il y a donc $\frac{n(n-1)}{2} + 1$ clauses.

    \subsubsection{Avec un encodage binaire}
    On peut faire avec $n log_2(n) + 1$ clauses (voir cours).

  \subsection{Problèmes de décision}
  On peut définir un problème de décision comme un langage de mots sur un alphabet
  fini $\Sigma$. On note $\Sigma^*$ l'ensemble des mots sur l'alphabet $\Sigma$, et
  $\epsilon$ le mot vide. Un langage sur $\Sigma$ est un sous-ensemble $L \subseteq \Sigma^*$. \\

  Un \textit{problème de décision} est un langage $P \subseteq \Sigma^*$.\\

  Chaque langage $P$ représente bien un problème dont la réponse est oui ou non, 
  en l'identifiant à sa fonction caractéristique $\chi_P$ :
  \begin{align}
    \chi_P : \Sigma^* & \rightarrow & \{0,1\} \\
    u & \mapsto & \begin{cases}1 \text{si} u \in P \\ 0 \text{si} u \not \in P\end{cases}
  \end{align}

  \subsection{Problème d'optimisation}
  Un \textit{problème d'optimisation} est un problème où l'on veut maximiser ou
  minimiser une certaine quantité. On peut associer un problème de décision à un
  problème d'optimisation, en donnant une borne. Si on sait résoudre le problème
  de décision associé à un problème d'optimisation, on peut parfois résoudre le
  problème d'optimisation.

  \subsection{Algorithme de décision}
  Un problème $P \subseteq \Sigma^*$ est décidé par un algorithme $A$ si pour 
  tout mot $u \in \Sigma^*$ : 
  \begin{itemize}
    \item $A$ termine et retourne 1 si $u \in P$
    \item $A$ termine et retourne 0 si $u \not \in P$
  \end{itemize}

  \subsection{La classe $\mathcal{P}$}
  La classe $\mathcal{P}$ est la classe des problèmes pouvant être décidés en temps
  polynomial. Plus précisément, un problème $P \subseteq \Sigma^*$ est dans $\mathcal{P}$
  si il existe un algorithme $A$ et une constante $k$ tel que pour tout mot $u$
  de longueur $n$, 
  \begin{itemize}
    \item $A$ retourne 1 en temps $O(n^k)$ si $u \in P$
    \item $A$ retourne 0 en temps $O(n^k)$ si $u \not \in P$
  \end{itemize}

  Exemples :
  \begin{itemize}
    \item décider si un tableau est trié
    \item décider si un entier codé en binaire est premier (difficile, problème
    resté ouvert pendant longtemps)
  \end{itemize}

  \subsection{Algorithme de vérification}
  Informellement, un algorithme de vérification est un algorithme qui, étant donnée
  une solution candidate à un problème de décision, décide si oui ou non cette
  solution est valide. Formellement : \\

  Un algorithme de vérification pour un problème $P \subseteq \Sigma^*$ est un
  algorithme $A$ prenant deux mots en argument, tel que 
  $$ P = \{ u \in \Sigma^* | \exists v \in \Sigma^*, A(u,v) =1 \} $$
  Lorsque $A(u,v) = 1$, $v$ est appelé un certificat pour $u$.

  \subsection{La classe $\mathcal{N}\mathcal{P}$}
  La classe $\mathcal{N}\mathcal{P}$ est la classe des problèmes pouvant être vérifiés
  en temps polynomial. Note : $\mathcal{P} \subseteq \mathcal{N}\mathcal{P} \subseteq$ ExpTime.\\

  La \textbf{grande conjecture} de l'informatique fondamentale est :
  $$ \mathcal{P} \neq \mathcal{N}\mathcal{P} $$

  Un problème de décision $P$ est $\mathcal{N}\mathcal{P}$-complet si il est dans
  $\mathcal{N}\mathcal{P}$ et tout autre problème $P'$ de $\mathcal{N}\mathcal{P}$
  se réduit à $P$ en temps polynomial.


% ---------------------------------------------------------------------------- %
\section{La logique des prédicats}
  En logique des prédicats,
  \begin{itemize}
    \item on ajoute les quantificateurs
    \item on généralise les valeurs que peuvent prendre les variables
    \item on ajoute des relations (appelées prédicats) pour décrire certaines
    relations entre ces valeurs
    \item on ajoute des symboles de fonction à la syntaxe
  \end{itemize}

  \paragraph{Exemple} $\forall x\forall y . PremierEntreEux(x,y) \leftrightarrow \exists x' \exists y' . x.x'+y.y' = 1$\\

  Les ingrédients pour construire les formules sont : connecteurs Booléens,
  quantificateurs, symboles de relations, de fonctions, constantes et termes. Pour
  satisfaire une formule, il faudra définir une interprétation des symboles dans un domaine.

  \subsection{Langages du premier ordre}
  Un \textit{langage $\mathcal{L}$} de la logique du premier ordre est caractérisé par
  \begin{itemize}
    \item{des symboles de relations ou prédicats, notés $p, q, r, s, ...$}
    \item{des symboles de fonctions, notés $f, g, h, ...$}
    \item{des symboles de constantes, notés $c, d, e, ...$}
  \end{itemize}
  \'A chaque prédicat $p$ (resp. fonction $f$), on associe un entier strictement
  positif appelé l'\textit{arité} de $p$ (resp $f$), c'est-à-dire le nombre
  d'arguments de $p$ (resp $f$). On notera parfois $p|_n$ $f|_n$.\\

  On utilise le prédicat "=" pour dénoter l'égalité. Si "=" fait partie du 
  vocabulaire du langage $\mathcal{L}$, on dit que $\mathcal{L}$ est égalitaire.

  \paragraph{Exemples} 
  $$\mathcal{L}_1 = \{ r|_1, c \}$$
  $$\mathcal{L}_2 = \{ r|_2, f|_1, g|_2, h|_2, c, d \}$$

  L'ensemble des \textit{termes d'un langage $\mathcal{L}$}, noté $\mathcal{T}$ est le plus petit
  ensemble qui contient les symboles de constantes et de variables et qui est
  clos par application des fonctions.

  \paragraph{Exemples}
  \begin{itemize}
    \item les seuls termes du langage $\mathcal{L}_1$ sont la constante $c$ et les variables
    \item les expressions suivantes sont des termes du langage $\mathcal{L}_2$ : $f(c)$, $f(h(f(c),d))$, $f(y)$,...
  \end{itemize}
  Un terme est \textit{clos} s'il est sans variable. $f(c)$ est clos.\\

  L'ensemble des \textit{formules atomiques} d'un langage $\mathcal{L}$ est l'ensemble
  des formules de la forme :
  \begin{itemize}
    \item $p(t_1, t_2, ..., t_n)$ où $p$ est un prédicat d'arité $n$ et $t_1, t_2, ..., t_n$ 
    sont des termes du langage $\mathcal{L}$
    \item $t_1 = t_2$ si $\mathcal{L}$ est égalitaire et $t_1, t_2$ sont des termes
    du langage $\mathcal{L}$
  \end{itemize}

  L'ensemble des \textit{formules du langage $\mathcal{L}$}, que l'on désigne par
  $\mathcal{F}(\mathcal{L})$ est défini par la grammaire suivante :
  $$ \phi ::= p(t_1, t_2,...,t_n) | \phi \land \phi | \phi \lor \phi | \lnot \phi | \phi \rightarrow \phi | \phi \leftrightarrow \phi | \exists x.\phi | \forall x.\phi | (\phi) $$

  Toute formule d'un langage du premier ordre se décompose de manière unique sous
  l'une, et une seule, des formes suivantes :
  \begin{itemize}
    \item une formule atomique
    \item $\lnot \phi$, où $\phi$ est une formule,
    \item $\phi \land \psi$, $\phi \lor \psi$, $\phi \rightarrow \psi$, $\phi \leftrightarrow \psi$
    où $\phi$ et $\psi$ sont des formules
    \item $\forall x. \phi$ ou $ \exists x.\phi$ où $\phi$ est une formule et $x$ une variable.
   \end{itemize}

   \paragraph{Occurences}
   Une \textit{occurence} d'une variable dans une formule est un couple constitué
   de cette variable et d'une place effective, c'est-à-dire qui ne suit pas 
   un quantificateur. \textbf{Exemple :} dans $r(x,z) \rightarrow \forall z.(r(y,z) \lor y = z)$,
   la variable $x$ possède une occurence, la variable $y$ deux et $z$ trois.

   Une occurence d'une variable $x$ dans une formule $\phi$ est une \textit{occurence libre}
   si elle ne se trouve dans aucune sous-formule de $\phi$, qui commence par une
   quantification $\forall x$ ou $\exists x$. Dans le cas contraire, l'occurence
   est dite \textit{liée}. Une variable est libre dans une formule si elle a 
   au moins une occurence libre dans cette formule. Une formule close est une formule
   sans variable libre.

   \subsubsection{Structures}
   Une \textit{structure $\mathcal{M}$} pour un langage $\mathcal{L}$ se compose
   d'un ensemble non vide $M$, appelé le \textit{domaine} et d'une interprétation des
   symboles de prédicats par des relations sur $M$, des symboles de fonctions
   par des fonctions de M, et des constantes par des éléments de M. Plus précisément:

   \begin{itemize}
     \item d'un sous ensemle de $M^n$, noté $r^{\mathcal{M}}$, pour chaque symbole
     de prédicat $r$ d'arité $n$ dans $\mathcal{L}$
     \item d'une fonction de $M^n$ dans $M$, notée $f^{\mathcal{M}}$, pour chaque symbole
     de fonction $f$ d'arité $m$ dans $\mathcal{L}$
     \item d'un élément de $M$, noté $r^{\mathcal{M}}$, pour chaque symbole de 
     constante $c$ dans $\mathcal{L}$
   \end{itemize}

   \paragraph{Exemple} L'ensemble des réels $\mathbb{R}$ permet de construire une 
   structure pour $\mathcal{L}_2 = \{r, f, g, h, c, d\}$ de la façon suivante:

   \begin{itemize}
     \item on interprète le prédicat $r$ comme l'ordre $\leq$ sur les réels
     \item on interprète $f$ comme la fonction $+1$, $g$ comme $+$ et $h$ comme $\times$
     \item on interprète les constantes $c$ et $d$ comme $0$ et $1$
   \end{itemize}
   Cette structure se note $$ \mathcal{M}_2 = (\mathbb{R}, \leq, +1, +, \times, 0, 1) $$ 


% ---------------------------------------------------------------------------- %
\section{Preuves de programmes}
  \subsection{Triplets de Hoare}
  Informellement, les triplets de Hoare sont de la forme $(\phi, P, \psi)$ où $P$
  est un programme, $\phi$ (resp $\psi$) est une pré-condition (resp. post-condition) 
  décrite par une formule de la logique du premier ordre. On notera en général $\llparenthesis \phi\rrparenthesis P \llparenthesis \psi\rrparenthesis $.
  Nous allons définir un système de preuve pour démontrer que, supposant que $\phi$
  est satisfaite, $\psi$ est satisfaite aussi après exécution du programme $P$.

  \subsection{Correction}
  Soit $\phi$ une condition (pré ou post) et $I:x\rightarrow \mathbb{Z}$ un état. 
  On dit que $I$ \textit{satisfait} $\phi$, noté $I \vDash \phi$, si $\mathbb{Z}, I \vDash \phi$.\\

  On dit qu'un triplet $\llparenthesis \phi\rrparenthesis P\llparenthesis \psi\rrparenthesis $ est \textit{partiellement satisfait} si
  pour tout état qui satisfait $\phi$, l'état résultant de l'exécution de $P$
  satisfait $\psi$, pourvu que $P$ termine. On note cette relation $\vDash_{par} \llparenthesis \phi\rrparenthesis P\llparenthesis \psi\rrparenthesis $.\\

  On dit qu'un triplet $\llparenthesis \phi\rrparenthesis P\llparenthesis \psi\rrparenthesis $ est \textit{satisfait} si
  pour tout état qui satisfait $\phi$, l'exécution de $P$ à partir de cet état termine
  et l'état résultant de cette exécution satisfait $\psi$. On note cette relation $\vDash \llparenthesis \phi\rrparenthesis P\llparenthesis \psi\rrparenthesis $.

  \subsection{Système de preuve $\vdash_{par}$}

  \paragraph{Composition}
  $$ \frac{\llparenthesis \phi\rrparenthesis C_1\llparenthesis \eta\rrparenthesis \quad \llparenthesis \eta\rrparenthesis C_2\llparenthesis \psi\rrparenthesis }{\llparenthesis \phi\rrparenthesis C_1;C_2 \llparenthesis \psi\rrparenthesis }Composition $$

  \paragraph{Assignation}
  $$ \frac{}{\llparenthesis \psi[E/x] \rrparenthesis x:=E \llparenthesis \psi \rrparenthesis}Assignation $$
  où $\psi[E/x]$ est la formule $\psi$ où on a remplacé toutes les occurences de $x$ par $E$.
  Cette règle est un axiome. \textbf{Exemple :} $\llparenthesis y+z \geq 0 \rrparenthesis x:= y+z \llparenthesis x \geq 0 \rrparenthesis$

  \paragraph{Si-Alors}
  $$  \frac{
        \llparenthesis \phi \land B \rrparenthesis C_1 \llparenthesis \psi \rrparenthesis \quad
        \llparenthesis \phi \land \lnot B \rrparenthesis C_2 \llparenthesis \psi \rrparenthesis
      }
      {
        \llparenthesis \phi \rrparenthesis\, \texttt{if}\, B\, \{ C_1 \}\, \texttt{else}\, \{ C_2\}\, \llparenthesis \psi \rrparenthesis
      } Si-Alors
  $$ 

  \paragraph{Implication}
  $$
    \frac{
      \phi' \vDash \phi \quad \llparenthesis\phi\rrparenthesis C \llparenthesis\psi\rrparenthesis \quad \psi \vDash \psi'
    }
    {
      \llparenthesis\phi'\rrparenthesis C \llparenthesis\psi'\rrparenthesis
    } Implication
  $$
  On pourra omettre une des implications si on n'a besoin que de l'une ou des deux.

  \paragraph{While Partiel}
  $$
    \frac{
      \llparenthesis\psi \land B \rrparenthesis C \llparenthesis\psi\rrparenthesis
    }
    {
      \llparenthesis\psi\rrparenthesis\, \texttt{while}\, B\, \{C\}\, \llparenthesis\psi \land \lnot B \rrparenthesis
    } WhilePar
  $$
  La propriété $\psi$ est appelée \textit{invariant}. Elle est toujours vraie à chaque
  passage dans la boucle, ainsi qu'après le dernier passage.

  \subsection{Correction et complétude}
  Soit $\llparenthesis\phi\rrparenthesis P \llparenthesis\psi\rrparenthesis$ un triplet
  de Hoare. Alors $\vDash_{par} \llparenthesis\phi\rrparenthesis P \llparenthesis\psi\rrparenthesis$
  si et seulement si $\vdash_{par} \llparenthesis\phi\rrparenthesis P \llparenthesis\psi\rrparenthesis$

  \subsection{Terminaison d'un programme}
  Comme l'instruction \texttt{while} est la seule source de non-terminaison,
  on va identifier une expression arithmétique $E$ et démontrer qu'elle
  décroît strictemenet à chaque passage dans la boucle. Comme la quantité ne peut
  être négative, on aura bien une preuve de terminaison. \\

  Une telle expression est appelée \textit{variant}.

  \paragraph{While Total}
  $$
    \frac{
      \llparenthesis\eta\land B \land 0 \leq E = e_0 \rrparenthesis
      C
      \llparenthesis\eta\land 0 \leq E < e_0 \rrparenthesis
    }
    {
      \llparenthesis\eta\land 0 \leq E = e_0 \rrparenthesis\,
      \texttt{while}\, B\, \{C\}\,
      \llparenthesis\eta\land\lnot B\rrparenthesis
    } WhileTot
  $$
  \begin{itemize}
    \item $E$ est l'expression arithmétique qui décroît à chaque exécution de $C$
    \item $e_0$ est une variable logique
    \item $\eta$ est l'invariant
  \end{itemize}


% ---------------------------------------------------------------------------- %
\section{Automates finis}
  Un automate fini lit une séquence de lettres de gauche à droite, possède un 
  nombre fini d'états. En fonction de l'état courant et de la lettre lue, il se
  déplace vers un autre état.\\

  L'état initial est représenté par une flèche sans source et les états finaux
  sont représentés par des doubles cercles. Le mot est accepté si et seulement si
  l'automate se trouve dans un état final à la fin du mot.

  \begin{figure}[H]
    \begin{center}
      \includegraphics[width=0.3\textwidth]{fsa/fsa1.eps}
      \caption{Automate fini $A$}
      \label{fig:fsa:1}
    \end{center}
  \end{figure}

  \paragraph{Langage accepté} Le \textit{langage accepté (ou reconnu)} par
  l'automate $A$ (fig. \ref{fig:fsa:1}), noté $L(A)$, est l'ensemble des mots
  qui contiennent un nombre pair de lettres $a$ :
  $$ L(A) = \{ w \in \{a,b\}^* | w\, \text{contient un nombre pair de} a \}$$

  \paragraph{Définition formelle} Un automate fini $A$ sur un alphabet $\Sigma$
  est un 4-uplet $(Q, q_0, F, \delta)$ où
  \begin{itemize}
    \item $Q$ est un ensemble fini d'éléments appelés \textit{états}
    \item $q_0$ est appelé \textit{état initial}
    \item $F \subseteq Q$ est un ensemble d'états dits \textit{finaux ou acceptants}
    \item $\delta : Q \times \Sigma \rightarrow Q $ est une fonction (pas nécessairement totale)
    appelée \textit{fonction de transition}
  \end{itemize}

  \paragraph{Exécution} Une \textit{exécution} de $A$ est une suite finie 
  $e = p_0 \sigma_1 p_1 \sigma_2 ... p_{n-1} \sigma_n p_n$ ($n \geq 0$) telle que
  \begin{itemize}
    \item $p_0 = q_0$
    \item $\forall i \in \{0,...,n\} : p_i \in Q$
    \item $\forall i \in \{1,...,n\} : \sigma_i \in \Sigma$
    \item $\forall i \in \{0,...,n-1\} : \delta(p_i, \sigma_{i+1})$ est définie
    et vaut $p_i+1$
  \end{itemize}
  On dit que l'exécution $e$ est \textit{acceptante} si l'état atteint est final ($p_n \in F$)

  \paragraph{Complétion} Un automate $A$ est dit \textit{complet} si sa fonction de
  transition est totale.\\

  \textbf{Lemme} On peut toujours transformer un automate $A$ en un automate $B$
  complet qui accepte le même langage. (Idée : ajouter un état supplémentaire 
  appelé état puits non final et ajouter les transition manquantes vers cet état)

  \subsection{Test du vide}
  Le problème VIDE est le suivant :
  \begin{itemize}
    \item \textbf{Entrée :} un automate $A$ sur un alphabet $\Sigma$
    \item \textbf{Sortie :} est-ce que $L(A) = \varnothing$ ?
  \end{itemize}

  \paragraph{Etats atteignables} Soit $A = (Q, q_0, F, \delta)$ un automate sur un
  alphabet $\Sigma$. Un état $q\in Q$ est dit \textit{atteignable} s'il existe
  un mot $w \in \Sigma^*$ et une exécution de $A$ sur $w$ qui se termine en $q$.\\

  Etant donné un automate $A$ avec $n$ états et $m$ transitions, on peut tester
  en $O(n+m)$ si $L(A) \neq \varnothing$ (en utilisant des algorithmes de graphe
  classiques comme le parcours en largeur par exemple).

  \subsection{Opérations Booléennes sur les langages}
  \paragraph{Complément de $L \subseteq \Sigma^*$ :} $\overline{L} = \{ w \in \Sigma^* | w \not \in L\} = \Sigma^* \setminus L$.

  \paragraph{Union, Intersection :} Trivial.

  \subsection{Clôture des automates par opérations Booléennes}
  Soient $A_1$ et $A_2$ des automates finis sur un alphabet $\Sigma$. Il existe
  des automates $A_c$, $U$ et $I$ tels que :
  \begin{center}
    \begin{align*}
      L(A_c) & = \overline{L(A_1)}\\
      L(U) & = L(A_1) \cup L(A_2)\\
      L(I) & = L(A_1) \cap L(A_2)
    \end{align*}
  \end{center}

  \paragraph{Clôture par complément}
  Si $A = (Q, Q_0, F, \delta)$ n'est pas complet, il faut le compléter, puis
  $A_c = (Q, Q_0, Q \setminus F, \delta)$. Par exemple, si $A = $

  \begin{figure}[H]
    \begin{center}
      \includegraphics[width=0.3\textwidth]{fsa/fsa1.eps}
    \end{center}
  \end{figure}

  alors $A_c =$

  \begin{figure}[H]
    \begin{center}
      \includegraphics[width=0.3\textwidth]{fsa/fsa1_complement.eps}
    \end{center}
  \end{figure}

  \paragraph{Produit d'automates}
  On appellera \textit{pré-automate} sur $\Sigma$ un triplet $(Q, q_0, \delta)$
  où $Q$ est un ensemble fini, $q_0 \in Q$ et $\delta : Q \times \Sigma \rightarrow Q$
  est une fonction.\\

  Soient $A_1 = (Q_1, q_0^1, F_1, \delta_1$ et $A_2 = (Q_2, q_0^2, F_2, \delta_2$
  deux automates sur $\Sigma$. Le produit de $A_1$ et $A_2$, noté $A_1 \otimes A_2$
  est le pré-automate défini par : 
  $$ A_1 \otimes A_2 = (Q_1 \times Q_2, (q_0^1, q_0^2), \delta_{12})$$
  où, pour tout $(q_1, q_2) \in Q_1 \times Q_2$, pour tout $\sigma \in \Sigma$,

  \begin{center} 
    \begin{align*} 
     \delta((q_1, q_2), \sigma) = 
      \begin{cases}
        \text{indéfini} & \text{si}\, \delta_1(q_1, \sigma)\, \text{est indéfinie}\\
        \text{indéfini} & \text{si}\, \delta_2(q_2, \sigma)\, \text{est indéfinie}\\
        (\delta_1(q_1, \sigma), \delta_2(q_2, \sigma)) & \text{sinon}.
      \end{cases} 
    \end{align*} 
  \end{center} 

  \paragraph{Exemple} Voici un exemple

  \begin{figure}[H]
    \begin{center}
      \begin{minipage}{0.45\textwidth}
        \includegraphics{fsa/fsa_productA1.eps}
        \caption{$A_1$}
      \end{minipage}\hfill
      \begin{minipage}{0.45\textwidth}
        \includegraphics{fsa/fsa_productA2.eps}
        \caption{$A_2$}
      \end{minipage}
    \end{center}
  \end{figure}

  \begin{figure}[H]
    \begin{center}
      \includegraphics[width=0.5\textwidth]{fsa/fsa_product.eps}
      \caption{$A_1 \otimes A_2$}
    \end{center}
  \end{figure}
  

\end{document}
